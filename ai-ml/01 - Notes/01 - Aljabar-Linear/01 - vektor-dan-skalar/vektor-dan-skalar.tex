\documentclass[12pt]{article}
\usepackage[utf8]{inputenc}
\usepackage[english, indonesian]{babel}
\usepackage{amsmath}
\usepackage{amsfonts}
\usepackage{amssymb}
\usepackage{graphicx}
\usepackage{geometry}
\usepackage{listings}
\usepackage{xcolor}
\usepackage{tcolorbox}
\usepackage{enumitem}
\usepackage{fancyhdr}

% Setup page geometry
\geometry{a4paper, margin=0.8in}

% Setup header and footer
\setlength{\headheight}{15pt}
\pagestyle{fancy}
\fancyhf{}
\rhead{\small Aljabar Linear: Vektor dan Skalar}
\lhead{\small Catatan Pembelajaran}
\rfoot{Halaman \thepage}
\renewcommand{\headrulewidth}{0.4pt}

% Setup for boxes to highlight definitions
\newtcolorbox{definitionbox}[1][]{
    colback=blue!5!white,
    colframe=blue!75!black,
    title=#1,
    fonttitle=\bfseries
}

\newtcolorbox{examplebox}[1][]{
    colback=green!5!white,
    colframe=green!75!black,
    title=#1,
    fonttitle=\bfseries
}

% Setup for code listings
\definecolor{codegreen}{rgb}{0,0.6,0}
\definecolor{codegray}{rgb}{0.5,0.5,0.5}
\definecolor{codepurple}{rgb}{0.58,0,0.82}
\definecolor{backcolour}{rgb}{0.95,0.95,0.92}

\lstdefinestyle{mystyle}{
    backgroundcolor=\color{backcolour},
    commentstyle=\color{codegreen},
    keywordstyle=\color{magenta},
    numberstyle=\tiny\color{codegray},
    stringstyle=\color{codepurple},
    basicstyle=\ttfamily\footnotesize,
    breakatwhitespace=false,
    breaklines=true,
    captionpos=b,
    keepspaces=true,
    numbers=left,
    numbersep=5pt,
    showspaces=false,
    showstringspaces=false,
    showtabs=false,
    tabsize=2
}

\lstset{style=mystyle}

\title{\textbf{Vektor dan Skalar dalam Aljabar Linear} \\ \large Catatan Pembelajaran Mendalam}
\author{Catatan Pembelajaran}
\date{\today}

\begin{document}

\maketitle

\begin{abstract}
Aljabar linear adalah dasar fundamental dalam ilmu komputer, khususnya dalam Machine Learning dan Artificial Intelligence. Vektor dan skalar adalah elemen dasar yang harus dipahami sebelum melanjutkan ke konsep-konsep lebih kompleks. Dokumen ini menyajikan penjelasan komprehensif tentang vektor dan skalar dengan contoh dan aplikasi praktis.
\end{abstract}

\tableofcontents
\newpage

\section{Pendahuluan}

Aljabar linear adalah cabang matematika yang sangat penting dalam ilmu komputer, khususnya dalam bidang Machine Learning dan Artificial Intelligence. Konsep dasar seperti vektor dan skalar merupakan fondasi utama yang harus dipahami sebelum melangkah ke konsep-konsep yang lebih kompleks seperti matriks, ruang vektor, dan transformasi linear.

Dokumen ini akan membahas secara mendalam tentang vektor dan skalar, disertai dengan contoh dan latihan untuk memperkuat pemahaman.

\section{Pengertian Dasar}

\subsection{Skalar}
\begin{definitionbox}[Definisi Skalar]
Skalar adalah besaran yang hanya memiliki \textbf{nilai} atau \textbf{magnitude} saja, tanpa arah. Skalar merupakan bilangan tunggal yang digunakan untuk merepresentasikan kuantitas tertentu seperti panjang, waktu, suhu, atau massa.
\end{definitionbox}

\subsubsection{Contoh Skalar}
\begin{itemize}[noitemsep]
    \item Suhu ruangan: $25^\circ$C
    \item Tinggi badan: 175 cm
    \item Waktu perjalanan: 2 jam
    \item Berat benda: 50 kg
    \item Harga: Rp 150.000
\end{itemize}

Dalam konteks aljabar linear dan komputasi, skalar biasanya direpresentasikan sebagai bilangan tunggal:
\begin{equation}
    s \in \mathbb{R}
\end{equation}
di mana $\mathbb{R}$ adalah himpunan bilangan real.

\subsection{Vektor}
\begin{definitionbox}[Definisi Vektor]
Vektor adalah besaran yang memiliki \textbf{nilai} (magnitude) dan \textbf{arah}. Vektor sangat penting dalam ilmu komputer dan Machine Learning karena digunakan untuk merepresentasikan data, fitur, dan parameter model.
\end{definitionbox}

\subsubsection{Contoh Vektor}
\begin{itemize}[noitemsep]
    \item Kecepatan mobil: 60 km/jam ke arah timur
    \item Koordinat dalam ruang: $(3, 4)$ atau $(1, 2, 5)$
    \item Fitur gambar dalam Machine Learning
    \item Parameter dalam model regresi
    \item Warna RGB: $(255, 0, 128)$
\end{itemize}

\subsection{Representasi Vektor}

Secara matematis, vektor biasanya ditulis dalam bentuk kolom:
\begin{equation}
    \mathbf{v} = \begin{bmatrix}
        v_1 \\
        v_2 \\
        \vdots \\
        v_n
    \end{bmatrix}
\end{equation}

atau dalam bentuk baris:
\begin{equation}
    \mathbf{v} = \begin{bmatrix}
        v_1 & v_2 & \cdots & v_n
    \end{bmatrix}
\end{equation}

Kita juga sering melambangkan vektor dengan huruf kecil tebal:
\begin{equation}
    \mathbf{v} = \begin{bmatrix}
        v_1 \\
        v_2 \\
        \vdots \\
        v_n
    \end{bmatrix}
\end{equation}

\begin{examplebox}[Contoh Vektor 2D dan 3D]
\textbf{Contoh vektor 2D:}
\begin{equation}
    \mathbf{v} = \begin{bmatrix}
        3 \\
        4
    \end{bmatrix}
\end{equation}

\textbf{Contoh vektor 3D:}
\begin{equation}
    \mathbf{w} = \begin{bmatrix}
        1 \\
        2 \\
        5
    \end{bmatrix}
\end{equation}
\end{examplebox}

\section{Operasi pada Vektor dan Skalar}

\subsection{Penjumlahan dan Pengurangan}

\subsubsection{Operasi antara Vektor dan Skalar}

Kita dapat menambahkan atau mengurangkan skalar ke setiap elemen vektor. Misalnya, jika:
\begin{equation}
    \mathbf{v} = \begin{bmatrix}
        v_1 \\
        v_2 \\
        \vdots \\
        v_n
    \end{bmatrix}
\end{equation}

dan $s$ adalah skalar, maka:
\begin{align}
    \mathbf{v} + s &= \begin{bmatrix}
        v_1 + s \\
        v_2 + s \\
        \vdots \\
        v_n + s
    \end{bmatrix} \\
    \mathbf{v} - s &= \begin{bmatrix}
        v_1 - s \\
        v_2 - s \\
        \vdots \\
        v_n - s
    \end{bmatrix}
\end{align}

\begin{examplebox}[Contoh Operasi Vektor dan Skalar]
\begin{equation}
    \mathbf{v} = \begin{bmatrix}
        1 \\
        2 \\
        3
    \end{bmatrix}, \quad s = 5
\end{equation}

\begin{equation}
    \mathbf{v} + s = \begin{bmatrix}
        1 + 5 \\
        2 + 5 \\
        3 + 5
    \end{bmatrix} = \begin{bmatrix}
        6 \\
        7 \\
        8
    \end{bmatrix}
\end{equation}
\end{examplebox}

\subsubsection{Operasi antara Dua Vektor}

Dua vektor dengan dimensi yang sama dapat dijumlahkan atau dikurangkan secara elemen per elemen:

\begin{definitionbox}[Penjumlahan dan Pengurangan Vektor]
Jika:
\begin{equation}
    \mathbf{u} = \begin{bmatrix}
        u_1 \\
        u_2 \\
        \vdots \\
        u_n
    \end{bmatrix}, \quad
    \mathbf{v} = \begin{bmatrix}
        v_1 \\
        v_2 \\
        \vdots \\
        v_n
    \end{bmatrix}
\end{equation}

Maka:
\begin{align}
    \mathbf{u} + \mathbf{v} &= \begin{bmatrix}
        u_1 + v_1 \\
        u_2 + v_2 \\
        \vdots \\
        u_n + v_n
    \end{bmatrix} \\
    \mathbf{u} - \mathbf{v} &= \begin{bmatrix}
        u_1 - v_1 \\
        u_2 - v_2 \\
        \vdots \\
        u_n - v_n
    \end{bmatrix}
\end{align}
\end{definitionbox}

\begin{examplebox}[Contoh Operasi Dua Vektor]
\begin{equation}
    \mathbf{u} = \begin{bmatrix}
        1 \\
        2 \\
        3
    \end{bmatrix}, \quad
    \mathbf{v} = \begin{bmatrix}
        4 \\
        5 \\
        6
    \end{bmatrix}
\end{equation}

\begin{equation}
    \mathbf{u} + \mathbf{v} = \begin{bmatrix}
        1 + 4 \\
        2 + 5 \\
        3 + 6
    \end{bmatrix} = \begin{bmatrix}
        5 \\
        7 \\
        9
    \end{bmatrix}
\end{equation}
\end{examplebox}

\subsection{Perkalian}

\subsubsection{Perkalian Skalar dengan Vektor}

\begin{definitionbox}[Perkalian Skalar dengan Vektor]
Ketika kita mengalikan sebuah vektor dengan skalar, kita mengalikan setiap elemen vektor dengan skalar tersebut:
\begin{equation}
    s \cdot \mathbf{v} = s \cdot \begin{bmatrix}
        v_1 \\
        v_2 \\
        \vdots \\
        v_n
    \end{bmatrix} = \begin{bmatrix}
        s \cdot v_1 \\
        s \cdot v_2 \\
        \vdots \\
        s \cdot v_n
    \end{bmatrix}
\end{equation}
\end{definitionbox}

\begin{examplebox}[Contoh Perkalian Skalar dengan Vektor]
\begin{equation}
    \mathbf{v} = \begin{bmatrix}
        2 \\
        3 \\
        1
    \end{bmatrix}, \quad s = 4
\end{equation}

\begin{equation}
    s \cdot \mathbf{v} = 4 \cdot \begin{bmatrix}
        2 \\
        3 \\
        1
    \end{bmatrix} = \begin{bmatrix}
        8 \\
        12 \\
        4
    \end{bmatrix}
\end{equation}
\end{examplebox}

\subsubsection{Perkalian Dua Vektor}

Ada beberapa jenis perkalian antara dua vektor. Dua yang paling umum adalah:

\begin{enumerate}[noitemsep]
    \item \textbf{Dot Product (Perkalian Titik)}: Menghasilkan skalar
    \item \textbf{Cross Product (Perkalian Silang)}: Menghasilkan vektor (hanya dalam 3 dimensi)
\end{enumerate}

\paragraph{Dot Product}

\begin{definitionbox}[Dot Product]
Dot product dari dua vektor dengan dimensi yang sama didefinisikan sebagai:
\begin{equation}
    \mathbf{u} \cdot \mathbf{v} = \sum_{i=1}^{n} u_i \cdot v_i = u_1v_1 + u_2v_2 + \cdots + u_nv_n
\end{equation}
\end{definitionbox}

\begin{examplebox}[Contoh Dot Product]
\begin{equation}
    \mathbf{u} = \begin{bmatrix}
        1 \\
        2 \\
        3
    \end{bmatrix}, \quad
    \mathbf{v} = \begin{bmatrix}
        4 \\
        5 \\
        6
    \end{bmatrix}
\end{equation}

\begin{equation}
    \mathbf{u} \cdot \mathbf{v} = (1 \cdot 4) + (2 \cdot 5) + (3 \cdot 6) = 4 + 10 + 18 = 32
\end{equation}
\end{examplebox}

\paragraph{Cross Product (hanya 3 dimensi)}

\begin{definitionbox}[Cross Product (3D)]
Cross product dari dua vektor 3 dimensi didefinisikan sebagai:

\begin{equation}
    \mathbf{u} \times \mathbf{v} = \begin{bmatrix}
        u_2v_3 - u_3v_2 \\
        u_3v_1 - u_1v_3 \\
        u_1v_2 - u_2v_1
    \end{bmatrix}
\end{equation}
\end{definitionbox}

\begin{examplebox}[Contoh Cross Product]
\begin{equation}
    \mathbf{u} = \begin{bmatrix}
        1 \\
        2 \\
        3
    \end{bmatrix}, \quad
    \mathbf{v} = \begin{bmatrix}
        4 \\
        5 \\
        6
    \end{bmatrix}
\end{equation}

\begin{equation}
    \mathbf{u} \times \mathbf{v} = \begin{bmatrix}
        (2 \cdot 6) - (3 \cdot 5) \\
        (3 \cdot 4) - (1 \cdot 6) \\
        (1 \cdot 5) - (2 \cdot 4)
    \end{bmatrix} = \begin{bmatrix}
        -3 \\
        6 \\
        -3
    \end{bmatrix}
\end{equation}
\end{examplebox}

\section{Norma Vektor}

\begin{definitionbox}[Definisi Norma Vektor]
Norma (atau magnitude) dari sebuah vektor menggambarkan panjang atau ukuran vektor tersebut. Ada beberapa jenis norma, dengan norma paling umum adalah:
\end{definitionbox}

\subsection{Norma-2 (Euclidean Norm)}

\begin{definitionbox}[Norma-2 (Euclidean)]
Didefinisikan sebagai:
\begin{equation}
    \|\mathbf{v}\|_2 = \sqrt{\sum_{i=1}^{n} v_i^2} = \sqrt{v_1^2 + v_2^2 + \cdots + v_n^2}
\end{equation}
\end{definitionbox}

\begin{examplebox}[Contoh Norma-2]
\begin{equation}
    \mathbf{v} = \begin{bmatrix}
        3 \\
        4
    \end{bmatrix}
\end{equation}

\begin{equation}
    \|\mathbf{v}\|_2 = \sqrt{3^2 + 4^2} = \sqrt{9 + 16} = \sqrt{25} = 5
\end{equation}
\end{examplebox}

\subsection{Norma-1}

\begin{definitionbox}[Norma-1]
Didefinisikan sebagai:
\begin{equation}
    \|\mathbf{v}\|_1 = \sum_{i=1}^{n} |v_i|
\end{equation}
\end{definitionbox}

\begin{examplebox}[Contoh Norma-1]
\begin{equation}
    \mathbf{v} = \begin{bmatrix}
        -3 \\
        4 \\
        -1
    \end{bmatrix}
\end{equation}

\begin{equation}
    \|\mathbf{v}\|_1 = |-3| + |4| + |-1| = 3 + 4 + 1 = 8
\end{equation}
\end{examplebox}

\subsection{Norma Tak Hingga (Max Norm)}

\begin{definitionbox}[Norma Tak Hingga]
Didefinisikan sebagai:
\begin{equation}
    \|\mathbf{v}\|_\infty = \max(|v_1|, |v_2|, \ldots, |v_n|)
\end{equation}
\end{definitionbox}

\begin{examplebox}[Contoh Norma Tak Hingga]
\begin{equation}
    \mathbf{v} = \begin{bmatrix}
        -3 \\
        4 \\
        -1
    \end{bmatrix}
\end{equation}

\begin{equation}
    \|\mathbf{v}\|_\infty = \max(|-3|, |4|, |-1|) = \max(3, 4, 1) = 4
\end{equation}
\end{examplebox}

\section{Aplikasi dalam Machine Learning}

\subsection{Representasi Data}

Dalam Machine Learning, data sering direpresentasikan sebagai vektor. Misalnya, dalam kasus pengenalan gambar, setiap piksel dalam gambar dapat menjadi elemen dari vektor. Jika kita memiliki gambar $28 \times 28$ piksel, maka gambar tersebut dapat direpresentasikan sebagai vektor dengan 784 elemen.

\begin{examplebox}[Contoh Representasi Data]
Untuk klasifikasi email spam:
\begin{equation}
    \mathbf{x} = \begin{bmatrix}
        \text{jumlah\_kata\_`free'} \\
        \text{jumlah\_kata\_`buy'} \\
        \text{panjang\_email} \\
        \text{jumlah\_karakter\_kapital} \\
        \vdots
    \end{bmatrix}
\end{equation}
\end{examplebox}

\subsection{Parameter Model}

Parameter dalam model Machine Learning juga sering disimpan dalam bentuk vektor. Misalnya, dalam model regresi linear:
\begin{equation}
    y = \mathbf{w}^T\mathbf{x} + b
\end{equation}
di mana $\mathbf{w}$ adalah vektor bobot (weights) dan $b$ adalah bias.

\section{Implementasi Python}

Berikut adalah contoh implementasi operasi vektor dan skalar menggunakan Python dan library NumPy:

\begin{lstlisting}[language=Python, caption=Operasi Vektor Menggunakan NumPy]
import numpy as np

# Membuat vektor
v = np.array([1, 2, 3])
u = np.array([4, 5, 6])
s = 5

# Operasi dasar
penjumlahan = v + u  # [5, 7, 9]
pengurangan = v - u  # [-3, -3, -3]
perkalian_skalar = s * v  # [5, 10, 15]
penambahan_skalar = v + s  # [6, 7, 8]

# Dot product
dot_product = np.dot(v, u)  # 32

# Norma vektor
norm_l2 = np.linalg.norm(v)  # 3.74 (akar dari 1^2 + 2^2 + 3^2)
norm_l1 = np.linalg.norm(v, ord=1)  # 6 (|1| + |2| + |3|)
\end{lstlisting}

\section{Sifat-Sifat Operasi Vektor}

\begin{definitionbox}{Sifat-Sifat Operasi Vektor}
Operasi vektor memiliki beberapa sifat penting:

\textbf{Sifat komutatif penjumlahan:} $\mathbf{u} + \mathbf{v} = \mathbf{v} + \mathbf{u}$

\textbf{Sifat asosiatif penjumlahan:} $(\mathbf{u} + \mathbf{v}) + \mathbf{w} = \mathbf{u} + (\mathbf{v} + \mathbf{w})$

\textbf{Elemen identitas penjumlahan:} $\mathbf{v} + \mathbf{0} = \mathbf{v}$

\textbf{Negatif vektor:} $\mathbf{v} + (-\mathbf{v}) = \mathbf{0}$

\textbf{Sifat distributif terhadap perkalian skalar:} $c(\mathbf{u} + \mathbf{v}) = c\mathbf{u} + c\mathbf{v}$

\textbf{Sifat komutatif dot product:} $\mathbf{u} \cdot \mathbf{v} = \mathbf{v} \cdot \mathbf{u}$
\end{definitionbox}

\section{Latihan}

\subsection{Latihan Dasar}

\textbf{Latihan 1:} Diberikan:
\begin{equation}
    \mathbf{a} = \begin{bmatrix}
        2 \\
        -1 \\
        4
    \end{bmatrix}, \quad
    \mathbf{b} = \begin{bmatrix}
        -3 \\
        2 \\
        1
    \end{bmatrix}
\end{equation}

Hitung:
\begin{enumerate}[noitemsep]
    \item $\mathbf{a} + \mathbf{b}$
    \item $\mathbf{a} - \mathbf{b}$
    \item $3\mathbf{a}$
    \item $\mathbf{a} \cdot \mathbf{b}$
\end{enumerate}

\textbf{Jawaban:}
\begin{enumerate}[noitemsep]
    \item $\mathbf{a} + \mathbf{b} = \begin{bmatrix} 2 + (-3) \\ -1 + 2 \\ 4 + 1 \end{bmatrix} = \begin{bmatrix} -1 \\ 1 \\ 5 \end{bmatrix}$

    \item $\mathbf{a} - \mathbf{b} = \begin{bmatrix} 2 - (-3) \\ -1 - 2 \\ 4 - 1 \end{bmatrix} = \begin{bmatrix} 5 \\ -3 \\ 3 \end{bmatrix}$

    \item $3\mathbf{a} = \begin{bmatrix} 3 \cdot 2 \\ 3 \cdot (-1) \\ 3 \cdot 4 \end{bmatrix} = \begin{bmatrix} 6 \\ -3 \\ 12 \end{bmatrix}$

    \item $\mathbf{a} \cdot \mathbf{b} = (2 \cdot -3) + (-1 \cdot 2) + (4 \cdot 1) = -6 + (-2) + 4 = -4$
\end{enumerate}

\textbf{Latihan 2:} Diberikan vektor $\mathbf{v} = \begin{bmatrix} 1 \\ 2 \\ 2 \end{bmatrix}$, hitung norma-1, norma-2, dan norma tak hingga dari vektor tersebut.

\textbf{Jawaban:}
\begin{align}
    \|\mathbf{v}\|_1 &= |1| + |2| + |2| = 5 \\
    \|\mathbf{v}\|_2 &= \sqrt{1^2 + 2^2 + 2^2} = \sqrt{1 + 4 + 4} = \sqrt{9} = 3 \\
    \|\mathbf{v}\|_\infty &= \max(|1|, |2|, |2|) = 2
\end{align}

\section{Visualisasi Vektor (2D dan 3D)}

Vektor dapat divisualisasikan dalam ruang 2D atau 3D. Dalam ruang 2D, vektor digambarkan sebagai panah dari titik asal ke koordinat yang ditentukan.

Untuk vektor $\mathbf{v} = \begin{bmatrix} v_1 \\ v_2 \end{bmatrix}$, kita menggambarkan panah dari titik $(0,0)$ ke titik $(v_1, v_2)$.

Dalam ruang 3D, vektor $\mathbf{v} = \begin{bmatrix} v_1 \\ v_2 \\ v_3 \end{bmatrix}$ digambarkan sebagai panah dari titik $(0,0,0)$ ke titik $(v_1, v_2, v_3)$.

\section{Kesimpulan}

Vektor dan skalar adalah konsep dasar dalam aljabar linear yang memiliki peran penting dalam berbagai bidang ilmu, khususnya dalam Machine Learning dan Artificial Intelligence. Dengan memahami operasi dasar pada vektor dan skalar, termasuk penjumlahan, pengurangan, perkalian skalar, dot product, cross product, dan norma vektor, kita dapat membangun fondasi yang kuat untuk memahami konsep-konsep lebih lanjut dalam aljabar linear seperti matriks, transformasi linear, dan dekomposisi matriks.

Dalam konteks Machine Learning, representasi data sebagai vektor sangat penting karena memungkinkan algoritma untuk memproses informasi dengan cara matematis dan statistik yang efektif.

\end{document}